\section{Generate \& Test-Realisierung des N-Damen-Problems}
\label{anhangE}

\begin{lstlisting}[language=Haskell,mathescape,caption=Generate \& Test Realisierung des N-Damen-Problems]
-- Definition der Ungleichheit mit Hilfe von (=:=)
diff :: a -> a -> Success
diff x y = (x == y) =:= False

-- Definition eines "kleiner-gleich"-Constraints mit (=:=)
lessEqual :: a -> a -> Success
lessEqual x y = (x <= y) =:= True

-- Befindet sich die Dame auf dem Schachbrett?
posOnBoard :: Int -> Int -> Success
posOnBoard q n = lessEqual 1 q & lessEqual q n

-- Generiert alle m�glichen Platzierungen f�r n Damen 
-- auf einem nxn-Schachbrett
generate :: Int -> [Int] -> Success
generate n l = 
  l =:= genVars n & foldl (\s q -> posOnBoard q n & s) success l

-- Testet, ob keine Dame eine andere schlagen kann
test :: [Int] -> Success
test [] = success
test (q:qs) = safe q qs 1 & test qs

-- Testet, ob die Dame q von keiner anderen Dame schlagbar ist
safe :: Int -> [Int] -> Int -> Success
safe _ [] _ = success
safe q (q1:qs) p = no_attack q q1 p & safe q qs (p+1)

-- Testet, ob q1 und q2 sich nicht gegenseitig schlagen k�nnen
no_attack q1 q2 p = diff q1 q2 & diff q1 (q2+p) & diff q1 (q2-p)

-- Erzeugt n freie Variablen
genVars n = if n == 0 then [] else var : genVars (n-1) 
 where var free

-- Beispielaufruf: generate 4 l $\&$ test l where l free
\end{lstlisting}
