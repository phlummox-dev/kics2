\documentclass{scrartcl}

\title{Implementing Residuation in KiCS2}

\author{%
Bj{\"o}rn Peem{\"o}ller \\
Institut f{\"u}r Informatik, CAU Kiel, D-24098 Kiel, Germany \\
\texttt{bjp@informatik.uni-kiel.de}
}

\date{}

\begin{document}

\maketitle
\thispagestyle{empty}

\begin{abstract}
Residuation is a well-known technique for functional-logic programming
languages to delay the evaluation of functions until their arguments are
sufficiently instatiated.
However, the KiCS2 Curry compiler currently does not support residuation
but performs narrowing steps instead due to the internal representation
of expressions to be evaluated.
In this talk, we present a slightly extended version of the
constraint representation that allows the representation
of suspended computations and their delayed concurrent evaluation.
\end{abstract}

\end{document}
